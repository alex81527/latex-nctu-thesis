% Author: W. Alex Chen, Department of Computer Science, National Chiao Tung University, Hsinchu, Taiwan
% Email: alex81527@gmail.com
% Date: May 12, 2017
%
% If you're a beginner at latex, the following page is quite helpful:
% https://en.wikibooks.org/wiki/LaTeX
%
%%%%%%%%%%%%%%%%%%%%%%%%%%%%%%%%%%%%%%%%%%%%%%%%%%%%%%%%%%%%%%%%%%%%%%%%%
% Fill in all the required fileds, and your're ready to go
\newcommand{\CoverPage}{cover_page/cover.pdf}
\newcommand{\TitlePage}{title_page/title.pdf}
% Fill in the fields in chinese_abstract/ab_zh.tex, and abstract/ab.tex as well
\newcommand{\ChineseAbstract}{chinese_abstract/ab_zh}
\newcommand{\Abstract}{abstract/ab}
\newcommand{\Acknowledgement}{acknowledgement/ack}
\newcommand{\Body}{body/sample-body}
% Save your bibtex here
\newcommand{\Bib}{sample-bibliography}

% Thesis title, student name, and advisor name
\newcommand{\TitleEnglish}{NGPLS - A Non GPS Position Location System}
\newcommand{\TitleChinese}{一 個 非G P S之 定 位 系 統}
\newcommand{\StudentNameEnglish}{Da-Ming Chang}
\newcommand{\StudentNameChinese}{張大明}
\newcommand{\AdvisorNameEnglish}{Shyan-Ming Yuan}
\newcommand{\AdvisorNameChinese}{袁賢銘}

% Using the following fonts is recommended
\newcommand{\EnglishFont}{Times New Roman}
\newcommand{\ChineseFont}{DFKai-SB}

%%%%%%%%%%%%%%%%%%%%%%%%%%%%%%%%%%%%%%%%%%%%%%%%%%%%%%%%%%%%%%%%%%%%%%%%%

\documentclass[12pt,a4paper]{report}
\pagestyle{plain}
\setcounter{secnumdepth}{3}
\setcounter{tocdepth}{3}

\usepackage{wallpaper} % watermark 
\usepackage{fontspec}  % use system fonts 
\usepackage{xeCJK}     % CJK support
\usepackage[left=3cm,right=2cm,top=2.5cm,bottom=2.5cm]{geometry}
\usepackage{times}
\usepackage{graphicx}
\usepackage{url}
\usepackage{amssymb}
\usepackage{amsthm}
\usepackage{amsfonts}
\usepackage{amsmath}
\usepackage{array}
\usepackage{multirow}
\usepackage{setspace}
\usepackage{cite}
\usepackage{balance}
\usepackage{color}
\usepackage{pdfpages}
%\usepackage{subfigure}
\usepackage{xcolor,colortbl}
\usepackage{caption}
\usepackage{subcaption}
\usepackage{algorithmic}
\usepackage{algorithm}


% Auto line break for chinese letters
\XeTeXlinebreaklocale "zh" 
\XeTeXlinebreakskip = 0pt plus 1pt 

% Select fonts
% Ligatures refer to the replacement of two separate characters with a specially drawn
% glyph for functional or æsthetic reasons.
\setmainfont[Ligatures=TeX]{\EnglishFont} 
%AutoFakeBold設定粗體字要多粗
%AutoFakeSlant設定斜體字要多斜,範圍-0.999到0.999,負值為往左斜
\setCJKmainfont[Ligatures=TeX, AutoFakeBold=3, AutoFakeSlant=.4]{\ChineseFont} 


% copy from macros.tex
\newcommand{\Eq}[1]{Eq.~(\ref{#1})}
\newcommand{\Fig}[1]{Fig.~\ref{#1}}
\newcommand{\Chap}[1]{Chapter~\ref{#1}}
\newcommand{\Sec}[1]{Section~\ref{#1}}
\newcommand{\Tab}[1]{Table~\ref{#1}}
\newtheorem{thm}{Theorem}
\newcommand{\bthm}{\begin{thm}\begin{textit}}
\newcommand{\ethm}{\end{textit}\end{thm}}
\newtheorem{lema}{Lemma}
\newcommand{\bdla}{\begin{lema}\begin{textit}}
\newcommand{\edla}{\end{textit}\end{lema}}
\newtheorem{theorem}{Theorem}
\newtheorem{lemma}{Lemma}
\newtheorem{prop}{Proposition}
\newtheorem{defn}{Definition}


%\newfont{\chfont}{CFONTmk14}

\begin{document}

% Add NCTU watermark, scaled to 50% size
\CenterWallPaper{0.5}{watermark/nctu_logo.pdf} 
\setlength{\wpXoffset}{0 cm}
\setlength{\wpYoffset}{0 cm}


\includepdf[pages={-}]{\CoverPage}
\includepdf[pages={-}]{\TitlePage}

%The line spacing is the nominal vertical space between lines, baseline to baseline.  It is
%stored in the parameter '\baselineskip'.  The default '\baselineskip' for the Computer Modern 
%typeface is 1.2 times the '\fontsize'.
\baselineskip 25pt 
\pagebreak 
\pagenumbering{Roman}

%The '\addcontentsline'{EXT}{UNIT}{TEXT} command adds an entry to the specified list or table where:
%
%EXT
%The extension of the file on which information is to be written, typically one of: 
%'toc' (table of contents), 'lof' (list of figures), or 'lot' (list of tables).
%
%UNIT
%The name of the sectional unit being added, typically one of the following, matching the value of the EXT argument:
%'toc' The name of the sectional unit: 'part', 'chapter', 'section', 'subsection', 'subsubsection'.
%'lof' For the list of figures.
%'lot' For the list of tables.
%
%ENTRY
%The text of the entry.
\addcontentsline{toc}{chapter}{Abstract in Chinese} 
\include{\ChineseAbstract}

\addcontentsline{toc}{chapter}{Abstract}           
\include{\Abstract}
 
\addcontentsline{toc}{chapter}{Acknowledgement}     
\include{\Acknowledgement}

\addcontentsline{toc}{chapter}{Contents}            
\tableofcontents

\addcontentsline{toc}{chapter}{List of Figures}     
\listoffigures

\addcontentsline{toc}{chapter}{List of Tables}      
\listoftables
\pagebreak

\setcounter{page}{1} 
\pagenumbering{arabic}
%\baselineskip 21pt

\include{\Body}

%\baselineskip=0pt
\parsep=-4pt
\itemsep=-4pt
\parskip=0pt

\bibliographystyle{abbrv}
\addcontentsline{toc}{chapter}{Bibliography}
\bibliography{\Bib}                        

%\baselineskip 14.5pt

\end{document}
