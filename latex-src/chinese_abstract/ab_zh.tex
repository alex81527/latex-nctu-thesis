%%%%%%%%%%%%%%%%%%%%%%%%%%%%%%%%%%%%%%%%%%%%%%%%%%%%%%%%%%%%%%%%%%%%%%%%%%%%%%%%%%%%%%%%%%%
% Fill in the required fileds: \AbstractInstituteChinese, \AbstractContentChinese, and 
% \AbstractKeywordsChinese

\newcommand{\AbstractInstituteChinese}{國立交通大學網路工程研究所}

\newcommand{\AbstractContentChinese}{
摘要(Abstract)又稱文摘或提要。 它是以簡明扼要的文句,將某種文獻的主要內容,正確無誤地摘錄出來,使讀者於最短的時間內,得知原著的大意。 
}
\newcommand{\AbstractKeywordsChinese}{
關鍵字1,關鍵字2,關鍵字3.
}
%%%%%%%%%%%%%%%%%%%%%%%%%%%%%%%%%%%%%%%%%%%%%%%%%%%%%%%%%%%%%%%%%%%%%%%%%%%%%%%%%%%%%%%%%%%

\begin{center}
	\begin{minipage}{\textwidth}
    \centering
	%Thesis title
    {\large \textbf{\TitleChinese}\par}
	\end{minipage}
\end{center}


\vspace{0.8cm}  
學生: \StudentNameChinese
\hfill
指導教授: \AdvisorNameChinese 教授

\vspace{0.5cm}
\centerline{\AbstractInstituteChinese}
\vspace{0.5cm}

%\centerline{\textbf{摘要}}

\begin{center}
	\begin{minipage}[s]{0.2\textwidth}
    \centering
	%Thesis title
    {\large \textbf{摘 \hfill 要}}
	\end{minipage}
\end{center}

 
\vspace{0.5cm}

\AbstractContentChinese
 
\vspace{1cm} 
{\bf 關鍵字:} \AbstractKeywordsChinese 
